\section{Change}
\label{sec:change}

\subsection{Behavior}

\Klang{} at JPL so far, has mostly been used for specifying static 
structure, similar to what can be represented by UML/SysML class diagrams, with 
requirements typically constraining integer and real variables that represent 
properties of physical nature, such as for example weights and distances.
The shapes model in Figure \ref{fig:shapes} is an example of a model of structure, namely the shape of triangles.
\Klang{} can, however, also be used for specifying behavior, using the same
concepts used for specifying structure, namely classes, properties, functions and
requirements on these. The idea is in other words to use mathematical logic to 
represent behaviors. This is an illustration of the pursued objective during 
the design of \Klang{} to keep the language as small as possible, relying as much 
as possible on the language of mathematics for expressing problems and solutions. 
This approach of course has algorithmic consequences when it comes to analyzing 
models. Our intention is to stay in mathematics as far as the tooling (existing 
theorem provers) allows us. In the following subsections we illustrate how one can 
encode two different behavior concepts in \Klang{}, namely state machines and event 
scheduling.

\subsubsection{State Machines}

State machines are commonly used to specify the behavior of software
as well as hardware systems. They are frequently used at JPL for
specifying the behavior of embedded flight software modules
controlling, for example, planetary rovers. A state machine is defined
by a collection of states, a collection of events, and a labeled
transition relation (labels are events) between states. This can of
course be modeled in numerous ways. Here we shall assume deterministic
state machines, and model the transition relation as a function.  The
\Klang{} model in Figure \ref{fig:statemachine} represents an encoding
of a state machine modeling a rocket engine, which can be in one of
the states: \name{off}, \name{ready} or \name{firing}. Events include
\name{turn\_on}, \name{fire}, and \name{turn\_off}. The types of
states and events (\name{State} and \name{Event}) are modeled as
body-less classes. The class \name{RocketEngine} models our state
machine. It defines the three states as well as the three events as
properties of the appropriate types. A requirement expresses that the
states are all different (a similar requirement should in principle
also be provided for events).

The function \name{move} represents the transition relation, and is
declared to take two arguments: a state and event, and to return a
state.  It has no body, meaning that it yet to be defined. The
subsequent four requirements define the \name{move} function. For
example the first requirement states that in the \name{off} state, on
encountering a \name{turn\_on} event, the engine moves to the
\name{ready} state.

The last requirement demanding the engine to move from the
\name{ready} state to the \name{off} state on a \name{fire} event was
added on a Monday morning by a tired modeler, and in our context
represents a change to the model.  This requirement, however, is
inconsistent with a previous requirement that demands the resulting
state to be \name{firing}.  Since \name{move} is a function (the
transition relation is deterministic), and cannot return two different
values for the same argument, this is detected by the solver.  Without
this last requirement, the solver will declare the model satisfiable,
and will synthesize the state machine function based on the provided
requirements.  Note, however, that not all transitions are modeled,
hence the synthesized state machine may not be the desired one.

\begin{figure}
\centering
\begin{tabular}{c}
\hline \\
\lstinputlisting{examples/statemachine.k} \\ \\
\hline
\end{tabular}
\caption{State machine}
\label{fig:statemachine}
\end{figure}

\subsubsection{Event Scheduling}

Event scheduling is a very common problem faced in a plethora of
fields and domains. The typical scenario usually involves specifying
multiple different allowed orderings of events and determining whether
the specified ordering is satisfiable or not, and if satisfiable,
possibly generating a timeline for the specified events.
Figure~\ref{fig:scheduling} shows one such example encoded using
\Klang{}.

\begin{figure}
\centering
\begin{tabular}{c}
\hline \\
\lstinputlisting{examples/scheduling.k} \\ \\
\hline
\end{tabular}
\caption{Scheduling}
\label{fig:scheduling}
\end{figure}

The foundation of the scheduling problem is specified by the
\name{Event} class, which represents a single event that has a start
and end time (both of these specified using the \code{Int} type). The
requirements on this class specify that the duration of any event has
to be non-zero (\code{req} \name{nonZeroDuration}) and the start time
of all events is greater than or equal to zero (\code{req}
\name{afterBigBang}). Further, the three functions in the \name{Event}
class encode Allen logic operators \cite{allen-logic-84} 
using simple mathematical
expressions on the start and end times of two events in
question. The functions specify whether an event occurs
\name{before} another event $e$, \name{meets} the other event, or
\name{contains} the other event. We do not include the full set of
Allen logic operators for sake of brevity. 

Using this foundation, it is now easy to specify schedules and
events. This is exactly what the \name{Schedule} class does. Five
events are created as instances of the \name{Event} class. The actual
schedule is specified as a requirement, which expresses the ordering
of the events. In this particular example, the schedule specifies that
the \name{sciWindow} (science window) must occur before the
\name{commWindow} (communication window) and at least one science
activity must have taken place. By translating the scheduling
specification to SMT2 and applying the \zthree{} theorem prover, it
can now be checked if the provided ordering on the events is
satisfiable or not by checking if the \name{Schedule} class is
satisfiable. A satisfying assignment to the class also provides us
with a concrete time line of the events.  It can be common for
schedules and events to change as the project evolves and reaches
maturity. Using such a mechanism for encoding the schedule of events
provides great power and flexibility. Dealing with the changing
schedule is easily done by either modifying existing constraints in
the \name{Schedule} class, or by adding new events and
constraints. Each change in the schedule is also verified using a
theorem prover, which adds greater confidence in the change.


\subsection{Refinement}

Change can be considered as refinement. In the formal methods literature
refinement usually refers to the situation where one model/program, the 
specification, and typically abstract of nature, is replaced by a lower 
level model/program, the implementation. Along with the refinement normally
goes a proof, that the implementation refines the specification.
The literature offers many solutions to how specifications, implementations and 
refinements are expressed as well as proved correct, see for example 
\cite{vdm,raise92,george-raise-2008}. We shall not here enumerate all of these, but bring forward two examples, one illustrating function refinement, and one illustrating data refinement.

\subsubsection{Function Refinement}

Function refinement consists of making the body of a function more concrete,
while the signature (name as well as  argument and result types) of the 
function stays unchanged. More generally, data 
structures accessed by the function stay unchanged. One popular approach to this
is design-by-contract, where a function is first specified using pre/post 
conditions, and then later implemented with a function body. This form of 
refinement is advocated for example in specification languages such as 
\vdm{} \cite{vdm} and \raiselang{} \cite{raise92,george-raise-2008}, as well as in programming languages such as \eiffel{} \cite{eiffel} and \java{} in the form of the \jml{} comment language \cite{jml}. 

\Klang{} supports design-by-contract using pre/post conditions. The example in
Figure \ref{fig:prepost} illustrates this with two class definitions. The left-most
class \name{Util\_Spec} represents the specification of a mathematics utility 
module containing two functions, \name{min} for computing the minimum of of
two values, and \name{abs} for returning the absolute value of an integer. Both 
functions are specified with a post condition stating what is expected to be true
about the resulting value, denoted by \name{\$result}. As an example, the post
condition for the \name{min} function states that the result is equal to one of
the arguments, and it is smallest such.

The class \name{Util}
to the right extends class \name{Util\_Spec} and refines the functions with
proper function bodies. The semantics of \Klang{} is such that the refined function 
bodies will have to satisfy the post conditions. The \Klang{} solver proves this
automatically in this case. The fact that the implementation class extends 
(inherits from) the specification class reflects that this form of refinement is a 
form of theory refinement, where the theory denoted by the implementation must imply 
that of the specification: the implementation signature contains that of the 
specification, and the requirements logically imply those of the specification:

\[
Imlementation \Rightarrow Specification
\]

\begin{figure}
  \centering
  \begin{tabular}[c]{c|c}
    \hline \\
    Specification & Implementation \\
    \hline\hline \\ \\
    \begin{subfigure}[c]{0.5\textwidth}
     \lstinputlisting[lastline=13]{examples/prepost.k}
      \label{fig:prepost1}
    \end{subfigure}
    &
    \begin{subfigure}[c]{0.5\textwidth}
      \lstinputlisting[firstline=15,xleftmargin=3.0ex]{examples/prepost.k}
      \label{fig:prepost2}
    \end{subfigure}
    \\ \\
    \hline
  \end{tabular}    
  \caption{Mathematical function refinement}
  \label{fig:prepost}
\end{figure}

\subsubsection{Data Refinement}

Data refinement consists of changing the data structures used, which will
cause functions to change as well. Data refinement has for example been
advocated in the \vdm{} method \cite{vdm}, which is the approach we shall
illustrate here using \Klang. The approach consists of defining a specification 
and an implementation as follows. The specification consists of a type
$\Sigma_a$ of abstract states, as well as abstract operations $opn_a : \Sigma_a 
\rightarrow \Sigma_a$ on this state. The implementation consists of a type
$\Sigma_c$ of concrete states, as well as concrete operations $opn_c : \Sigma_c 
\rightarrow \Sigma_c$ on this state. To perform a proof of correctness of the 
refinement, an abstraction function $abs : \Sigma_c \rightarrow \Sigma_a$ from 
the type of concrete implementation states to the type of abstract specification 
states must be provided, and the following property must (amongst others) 
be proved for each operation $opn$, where $opn_a$ is the abstract version and $opn_c$ is the concrete version, and $pre_a : \Sigma_a \rightarrow \mathbb{B}$
is the pre-condition of the abstraction operation $opn_a$:

\begin{equation} \label{eq:refinement}
\forall \sigma : \Sigma_c \cdot 
  pre_a(abs(\sigma)) \Rightarrow opn_a(abs(\sigma)) = abs(opn_c(\sigma))
\end{equation}

\noindent
Each concrete operation must in other words be proved to update the concrete state
in a manner corresponding to the desired operation on the abstract state.
We illustrate this approach with a rather simple \Klang{} model of a light switch,
which can be turned on and off.  The specification is shown on the left of
Figure \ref{fig:lightswitch}. A state is defined abstractly as an object of 
a class \name{State}. The two states \name{off} and \name{on} are defined as 
distinct states of that type. Two functions are defined, one for toggling
the state (\name{toggle}), and one for testing whether the light switch is on
(\name{isOn}). The \name{toggle} function is only declared by its signature, no 
function body is provided. The behavior is instead provided  as a couple of
requirements.

\begin{figure}
  \centering
  \begin{tabular}[c]{c|c}
    \hline \\
    Specification & Implementation \\
    \hline\hline \\ \\
    \begin{subfigure}[c]{0.5\textwidth}
     \lstinputlisting[lastline=17]{examples/lightswitch.k}
      \label{fig:lightswitch1}
    \end{subfigure}
    &
    \begin{subfigure}[c]{0.5\textwidth}
      \lstinputlisting[firstline=19,lastline=27,xleftmargin=5.0ex]{examples/lightswitch.k}
      \label{fig:lightswicth2}
    \end{subfigure}
    \\ \\
    \hline
  \end{tabular}    
  \caption{Lightswitch refinement}
  \label{fig:lightswitch}
\end{figure}

The implementation is shown on the right of Figure \ref{fig:lightswitch}.
Here we have decided to model the state as an integer, being 1 when the light 
switch is on and 0 when it is off. Note that in this case the implementation
does not extend (inherit from) the specification as was the case in the 
mathematical function refinement in Figure \ref{fig:prepost}. Instead, the
proof corresponding to equation (\ref{eq:refinement}) above is provided
in the separate class \name{RefinementProof} in Figure \ref{fig:lightswitch-proof}.
To express the refinement property to be proved, an instance \name{spec} of the
specification and an instance \name{impl} are created such that we can refer to their respective operations (functions). Then the abstraction function \name{abs} is defined from the concrete state of integers to the abstract state \name{State}
of the specification. Finally, the requirement is the \Klang{} formulation of 
equation (\ref{eq:refinement}) above, ignoring the pre-condition part since all
pre-conditions in this example are true. The \Klang{} solver proves the
implementation correct automatically. An incorrect modification of the implementation, such as for example to change the body of \name{isOn} in 
the implementation to $cs = 0$ will dually be caught by the solver.


\begin{figure}
\centering
\begin{tabular}{c}
\hline \\
\lstinputlisting[firstline=29]{examples/lightswitch.k} \\ \\
\hline
\end{tabular}
\caption{Lightswitch refinement proof}
\label{fig:lightswitch-proof}
\end{figure}
